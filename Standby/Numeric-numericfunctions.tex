%---------------------------%
\frametitle{Mathematical and Statistical Commands}

\frametitle{Useful Statistical Commands}
\begin{itemize}
\item \texttt{mean()} mean of a data set
\item \texttt{median()} median of a data set
\item \texttt{length()} Sample Size
\item \texttt{IQR()} Inter-Quartile Range
\item \texttt{var()} variance
\item \texttt{sd()} Standard Deviation
\item \texttt{range()} Range of a data set
\end{itemize}

\frametitle{useful operators}

Factorials
$n! = n \times n-1 \times \ldots \times 2 \times 1 $
Binomial Coefficients
\[ { n \choose k }  = \frac{n!}{(n-k)! \times k!}\]
%--------------------%
\frametitle{Managing Precision}

\begin{itemize}
\item \texttt{floor()} Floor function of x, $\lfloor x \rfloor$.
\item \texttt{ceiling()} Ceiling function of x, $\lceil x \rceil$.
\item \texttt{round()} Rounding a number to a specified number of decimal places.
\end{itemize}
%--------------------------------------------%
\frametitle{The Birthday function}
The R command pbirthday() computes the probability of a coincidence of a number of randomly chosen people sharing a birthday, given that there are n people to choose from.
Suppose there are four people in a room. The probability of two of them sharing a birthday is computed as about 1.6 \%
\begin{verbatim}
> pbirthday(4)
[1] 0.01635591
\end{verbatim}

How many people do you need for a greater than 50\% chance of a shared birthday? (choose from 23,43,63,83)?

%--------------------%
\frametitle{Set Theory Operations}
\begin{itemize}
\item \texttt{union()} union of sets A and B
\item \texttt{intersect()} intersection of sets A and B
\item \texttt{setdiff()} set difference A-B (order is important)
\end{itemize}

\begin{framed}
\begin{verbatim}
x = 5:10
y = 8:12
union(x,y)
intersect(x,y)
setdiff(x,y)
setdiff(y,x)
\end{verbatim}
\end{framed}

