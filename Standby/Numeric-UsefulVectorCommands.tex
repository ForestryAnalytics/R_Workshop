%numeric
%
%Creates or coerces objects of type "numeric".
%numeric(length = 0)
%as.numeric(x, ...)
%is.numeric(x)
%------------------------------------------%
\frametitle{Useful Commands For Vectors}

\begin{verbatim}
Newvec = c(13,16,36,55,23,11)
sort(Newvec)
rev(Newvec)
rep(Newvec,2)
rep(Newvec,3)
rep(Newvec,each=3)
diff(Newvec)
order(Newvec)
rank(Newvec)
\end{verbatim}

\begin{framed}
\begin{verbatim}
> Newvec = c(13,16,36,55,23,11)
>
> sort(Newvec)
[1] 11 13 16 23 36 55
> rev(Newvec)
[1] 11 23 55 36 16 13
>
> rep(Newvec,2)
 [1] 13 16 36 55 23 11 13 16 36 55 23 11
> rep(Newvec,3)
 [1] 13 16 36 55 23 11 13 16 36 55 23 11 13 16 36 55 23 11
>
> rep(Newvec,each=3)
 [1] 13 13 13 16 16 16 36 36 36 55 55 55 23 23 23 11 11 11
> diff(Newvec)
[1]   3  20  19 -32 -12
> order(Newvec)
[1] 6 1 2 5 3 4
> rank(Newvec)
[1] 2 3 5 6 4 1
\end{verbatim}
\end{framed}
%------------------------------------------------------%
\frametitle{Sequences}
\frametitle{Using the colon operator}
A `count-up' or a `count-down' will be determined automatically.
\begin{verbatim}
1:20
20:1
10:20
\end{verbatim}
\frametitle{Using the \texttt{seq()} operator}
Firstly we will mimic the sequences that we have created using the colon operator.
\begin{verbatim}
seq(1,20)
seq(20,1)
\end{verbatim}
