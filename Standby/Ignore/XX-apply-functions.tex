%------------------------------------------------------------------------------------------APPLY FAMILY--%
\frametitle{The \texttt{apply()} family of functions}

The "apply" family of functions keep you from having to write loops to perform 
some operation on every row or every column of a matrix or data frame, or on 
every element in a list.

The \texttt{apply()} function
The \texttt{apply()} function is a powerful device that operates on arrays and,
 in particular, matrices.
The \texttt{apply()} function returns a vector (or array or list of values) 
obtained by applying a specified function to either the row or columns of 
an array or matrix.
To specify use for rows or columns, use the additional argument of 1 for rows, 
and 2 for columns.

\begin{verbatim}
# create a matrix of 10 rows x 2 columns
m <- matrix(c(1:10, 11:20), nrow = 10, ncol = 2)

# mean of the rows

apply(m, 1, mean)
# [1]  6  7  8  9 10 11 12 13 14 15

# mean of the columns
apply(m, 2, mean)
#[1]  5.5 15.5
\end{verbatim}
The local version of apply()is lapply(), which computes a function for each 
argument of a list., provided each argument is compatible with the function argument (e.g. that is numeric).

The lapply() command returns a list of the same length as a list �X�, each 
element of which is the result of applying a specified function to 
the corresponding element of X.

A user friendly version of lapply() is sapply().

The sapply() command  is a variant of lapply() � returning a matrix 
instead of a list - again of the same length as a list X, 
each element of which is the result of applying a specified function to the
 corresponding element of X.
\begin{verbatim}
> x <- list(a=1:10, b=exp(-3:3), logic=c(T,F,F,T))
>
> # compute the list mean for each list element
>
> lapply(x,mean)
$a
[1] 5.5

$b
[1] 4.535125

$logic
[1] 0.5
>
> sapply(x,mean)
       a        b    logic
5.500000 4.535125 0.500000
>
\end{verbatim}





%-------------------------------------------------------------------------------------------------------------%
