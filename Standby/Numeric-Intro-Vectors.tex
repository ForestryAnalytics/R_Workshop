 \documentclass{beamer}
 
 \usepackage{amsmath}
 \usepackage{graphicx}
 
 \usepackage{framed}
 \usepackage{amssymb}
 
 \begin{document}
%---------------------------------------------------------------------------------%

\begin{frame}[fragile]{Vectors}
\begin{itemize}
\item $R$ operates on named data structures. The simplest such
structure is the vector, which is a single entity consisting of an
ordered collection of numbers or characters.

\item The most common types of vectors are:
\begin{itemize}
\item Numeric vectors \item Character vectors \item Logical
vectors
\end{itemize}

\item There are, of course, other types of vectors.
\begin{itemize}
\item Colour vectors - potentially useful later on.
\item Order vectors - The rankings of items in a vector.
\item Complex number vectors - not part of this course.
\end{itemize}
\end{itemize}
\end{frame}
%---------------------------------------------------------------------------------%
\begin{frame}[fragile]
\frametitle{Vectors: Creating and editing a vector}
\begin{itemize}
\item From last class.
\item To create a vector, use the assignment operator ``$=$" or ( $<-$ )and
the concatenate function ``c()". \item For numeric vectors, the values
entered are simply numbers.
\begin{framed}
\begin{verbatim}
>x =c(10.4,5.6,3.1,6.4,8.9)
>
\end{verbatim}
\end{framed}
% And, from last week, we can use the ``data.entry()" function to edit our vector.
% \begin{verbatim}
% >data.entry(x)
% >
% \end{verbatim}
\end{itemize}
\end{frame}



%---------------------------------------------------------------------------------%
\begin{frame}[fragile]
\frametitle{Vectors: Character \& logical vector}

\begin{itemize}
\item For character vectors, the values are simply characters,
specified with quotation marks.
\item Single quotation marks
\begin{framed}
\begin{verbatim}

Charvec <- c(`Dog', `Cat', `Shed', `Spoon')

\end{verbatim}
\end{framed}

\item A logical vectors is a vector whose elements are TRUE, FALSE
or NA (i.e. null)

\begin{framed}
\begin{verbatim}

Logvec <- c(TRUE, FALSE,TRUE,TRUE )

\end{verbatim}
\end{framed}

\end{itemize}
\end{frame}
%---------------------------------------------------------------------------------%
\begin{frame}[fragile]{Graphical Data Entry Interface}
\begin{itemize}

\item The \texttt{data.entry()} command calls a spreadsheet graphical user
interface, which can be used to edit data. All changes are saved
automatically.




\item Alternatively, the \texttt{edit()} command calls the `R editor',
which can be used to edit specified data or the code used to
define that data.

\begin{verbatim}
x<-edit(x)
\end{verbatim}

\end{itemize}
\end{frame}


%---------------------------------------------------------------------------------%

\begin{frame}[fragile]{Vectors: Empty vectors}

\begin{itemize}
\item Another method of creating vectors is to use the follow

\begin{itemize}
\item \texttt{numeric(length = n)} 
\item \texttt{character (length = n)} 
\item \texttt{logical (length = n)}
\end{itemize}

\item These commands create empty vectors, of the appropriate
kind, of length n.

\begin{framed}
\begin{verbatim}
> x<-numeric(4)
> x
[1] 0 0 0 0
\end{verbatim}
\end{framed}
%\item You can use the graphical data entry interface to populate your data sets.
\end{itemize}
\end{frame}

%---------------------------------------------------------------------------------%
\begin{frame}[fragile]{Vectors: Characteristics}

\begin{itemize}
\item We can use several $R$ commands to gather information about
a vector.

\begin{itemize}
\item \texttt{length(x)} - how many elements in a vector.  
\item \texttt{unique(x)} - display each unique item in a vector.  
\item \texttt{sum(x)} - the sum of the elements in a vector. 
\item \texttt{prod(x)} - the product of the elements in a vector.
\end{itemize}

\item We can also find statistical information about a vector
\begin{itemize}
\item \texttt{summary(x)} - summary statistics of a vector.  \item \texttt{mean(x)} -
the mean value of a vector. \item \texttt{sd(x)} -  the standard deviation
of a vector.
\end{itemize}

% \item Refer to the reference card for more commands to try out.
\end{itemize}
\end{frame}
%---------------------------------------------------------------------------------%
\begin{frame}[fragile]{Vectors: Characteristics (contd)}



\begin{framed}
\begin{verbatim}
> mean(x)
[1] 6.375
> sd(y)
[1] 2.858846
>
> median(z)
[1] 16
>
> summary(x)
   Min. 1st Qu.  Median    Mean 3rd Qu.    Max.
  3.100   4.975   6.000   6.375   7.400  10.400
\end{verbatim}
\end{framed}
\end{frame}

%---------------------------------------------------------------------------------%
\begin{frame}[fragile]{Calculations using vectors}

\begin{itemize}
\item Calculations are performed on a vector on a case-wise basis.
That is to say, the calculations are carried out on each element individually.
\begin{verbatim}
> y^2
[1]  2.56 12.25 60.84 44.89 65.61
\end{verbatim}

\begin{framed}
\begin{verbatim}
x <- c(11.1, 11.54,15.6,17.8,16.9,14.6, 12.7)
y <- c(0.2,    0.6, 0.7, 0.3,0.3,0.5,0.6)
z <- 1:3
\end{verbatim}
\end{framed}

\end{itemize}
\begin{itemize}
\item Try the following calculations.
\begin{verbatim}
> y*z
>
> sum(z)
>
> sum(y^2)
>
> sum(y*z)
>
\end{verbatim}
\end{itemize}
\end{frame}
%---------------------------------------------------------------------------------%
\begin{frame}[fragile]{Accessing vector's elements}

\begin{itemize}
\item The $n$th element of vector `x' can be accessed by
specifying its index when calling `x'.
\begin{verbatim}
>x[3]
[1] 15.6
\end{verbatim}

\item A sequence of  elements of vector `x' can be accessed by
specifying the lower and upper bound of the the range, in form
x[l:u].
\begin{verbatim}
> x[2:4]
\end{verbatim}
\end{itemize}
\end{frame}


%---------------------------------------------------------------------------------%

\begin{frame}[fragile]{Modifying a vector}

\begin{itemize}
\item A vector can be updated by assigning an extra value to it.
\begin{verbatim}
> logvec<-c(logvec,TRUE)
> logvec
[1]  TRUE FALSE  TRUE  TRUE  TRUE
\end{verbatim}

\item A vector can be repeated $n$ times using the \texttt{rep()} command.
\begin{verbatim}
> rep(charvec,2)
[1] "blue" "pink" "red" "blue" "pink" "red"
\end{verbatim}

\item Omitting and deleting the $n$th element of vector `x'.
\begin{verbatim}
>charvec[-5]
>charvec <- charvec[-5]
\end{verbatim}

\end{itemize}
\end{frame}
%---------------------------------------------------------------------------------%
\begin{frame}[fragile]{Relational operators}
A relational operator tests some kind of relation between two
entities. For $R$ the relational operators are as follows:
\begin{center}
\begin{tabular}{|c|c|c|c|}
  \hline

  Equals & == & Less or equal to  & <= \\
  \hline
  Not Equal & != & Greater than & > \\
  \hline
  Less than & < & Greater than & >= \\
  \hline
\end{tabular}
\end{center}
\end{frame}
%---------------------------------------------------------------------------------%
\begin{frame}[fragile]
\frametitle{Logical operators}
\begin{itemize}
\item The logical operators are AND, OR and NOT

\item if c1 and c2 are logical expressions, then $c1 \& c2$ is
their intersection (`AND'), $c1 | c2$ is their union (`OR'), and
$!c1$ is the negation of c1.
\end{itemize}
\begin{center}
\begin{tabular}{|c|c|c|c|}
  \hline
  AND & $ \& $ & also  & $\&\&$ \\
  \hline
  OR & $|$ & also & $||$ \\
  \hline
  NOT & $!$ & &  \\
  \hline
\end{tabular}
\end{center}
\end{frame}
%numeric
%
%Creates or coerces objects of type "numeric".
%numeric(length = 0)
%as.numeric(x, ...)
%is.numeric(x)
%------------------------------------------%
\begin{frame}[fragile]
\frametitle{Useful Commands For Vectors}

\begin{framed}
\begin{verbatim}
x  = c(13,16,36,55,23,11)

sort(x )
rev(x)
rep(x ,2)
rep(x ,3)
rep(x ,each=3)
diff(x )
order(x )
rank(x )
\end{verbatim}
\end{framed}
\end{frame}
%%%%%%%%%%%%%%%%%%%%%%%%%%%%%%%%%%%%%%%%%%%%%%%%%%%%%%%%%%
\begin{frame}[fragile]
\begin{framed}
\begin{verbatim}
> x  = c(13,16,36,55,23,11)
>
> sort(x )
[1] 11 13 16 23 36 55
>
> rev(x )
[1] 11 23 55 36 16 13
\end{verbatim}
\end{framed}
\end{frame}
%%%%%%%%%%%%%%%%%%%%%%%%%%%%%%%%%%%%%%%%%%%%%%%%%%%%%%%%%%
\begin{frame}[fragile]
\begin{framed}
\begin{verbatim}
> rep(x ,2)
 [1] 13 16 36 55 23 11 13 16 36 55 23 11
> rep(x ,3)
 [1] 13 16 36 55 23 11 13 16 36 55 23 11 13 16 36 55 23 11
>
> rep(x ,each=3)
 [1] 13 13 13 16 16 16 36 36 36 55 55 55 23 23 23 11 11 11
 \end{verbatim}
\end{framed}
\end{frame}
%%%%%%%%%%%%%%%%%%%%%%%%%%%%%%%%%%%%%%%%%%%%%%%%%%%%%%%%%%
\begin{frame}[fragile]
\begin{framed}
\begin{verbatim}
> diff(x )
[1]   3  20  19 -32 -12
>
> order(x )
[1] 6 1 2 5 3 4
>
> rank(x )
[1] 2 3 5 6 4 1
\end{verbatim}
\end{framed}
\end{verbatim}
\end{framed}
\end{frame}
%%%%%%%%%%%%%%%%%%%%%%%%%%%%%%%%%%%%%%%%%%%%%%%%%%%%%%%%%%
\begin{frame}[fragile]

\frametitle{Sequences}
\frametitle{Using the colon operator}
A `count-up' or a `count-down' will be determined automatically.
\begin{framed}
\begin{verbatim}
1:20
20:1
10:20
\end{verbatim}
\end{framed}
\end{frame}
%%%%%%%%%%%%%%%%%%%%%%%%%%%%%%%%%%%%%%%%%%%%%%%%%%%%%%%%%%
\begin{frame}[fragile]
\frametitle{Using the \texttt{seq()} operator}
Firstly we will mimic the sequences that we have created using the colon operator.
\begin{framed}
\begin{verbatim}
seq(1,20)
seq(20,1)
\end{verbatim}
\end{framed}
\end{frame}

%---------------------------------------------------------------------------------%
\begin{frame}[fragile]
\frametitle{Examples using operators}

We can use relational and logical operators to selecting elements
of a vector with specified criteria.

\begin{framed}
\begin{verbatim}
x <- 1:12

#selecting all elements of x greater than 5
x[x>5]

#selecting all elements of x greater or equal to than 5
x[x>=5]

#selecting all elements of x greater than 5 #or less than 3
x[(x>5)|(x<3)]

#selecting all elements of x between 3 and 5
x[(x>3)&(x<5)]
\end{verbatim}
\end{framed}
\end{frame}
%=====================================================================================================================%
\begin{frame}[fragile]
\frametitle{Data Selection and manipulation}

\begin{itemize}
\item \texttt{sort(x)} : sorts the object $x$ in ascending order.
\item \texttt{rev(x)} : reverses the order of $x$ without sorting it
\end{itemize}
\end{frame}
%---------------------------------------------------------------------------------%
\end{document}
